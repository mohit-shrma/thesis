\section{Introduction} \label{ch:matcomp:intro}

%Recommender systems are used in e-commerce, social networks, and web
%search to suggest the most relevant items to each user. 

Recommender systems
commonly use methods based on Collaborative Filtering~\cite{SarwarKarypis01},
which rely on the
historical preferences of the users over items in order to generate
recommendations. 
These methods predict the ratings for the items not rated by
the user and then select the items with the highest predicted ratings as 
item recommendations.
In \TOPN recommendations, $n$ unrated items with highest predicted ratings and
for small values of $n$, e.g., 10 and 50, are
served as recommendations.

In practice, the users do not provide their ratings to all the items, and hence
we observe only partial entries in the rating matrix. For the task of
recommendations, we need to complete the matrix by predicting the missing
ratings and select the unrated items with high predicted ratings as
recommendations for a user.
The matrix completion-based methods, discussed in Section~\ref{ch:related}, estimate the missing ratings based on the observed ratings in the matrix. These methods require entries in the matrix should be sampled uniformly at random for accurate recovery of the underlying low-rank model.
However, most real-world rating matrices exhibit a skewed distribution of
ratings as some users have provided ratings to few items and certain items have
received few ratings from the users. This skewed distribution may result in
insufficient ratings for certain users and items, and can thus affect the
accuracy of the matrix completion-based methods.

\iffalse
However, most real-world rating matrices do not satisfy the requirement of the
uniform random sampling as some users have provided ratings to few items and
certain items have received few ratings from the users; thereby leading to a
skewed distribution of ratings in the user-item rating matrix.
However, most real-world rating matrices do not satisfy the requirement of the
uniform random sampling as some users have provided ratings to few items and
certain items have received few ratings from the users; thereby leading to a
non-uniform and sparse rating matrix.
The potential issues associated with the fact that the observed entries are not
randomly distributed, have recently been studied in the context of
not-missing-at-random~\cite{Steck2010MNAR,Steck2013EvalRec} approaches and various methods have been developed that
incorporate a missing data model~\cite{lim2015top,KimRecsys2014} to improve the quality of the recommendations.
\fi


This chapter investigates how does the skewed distribution of the ratings in the
user-item rating matrix
affects the accuracy and the ranking performance of the matrix completion-based methods and shows that the
items having few ratings tend to have lower prediction accuracy. 
The key contributions of the work presented in this chapter are the following:
\begin{enumerate}
  \item shows that the skewed distribution of ratings in the user-item rating matrix 
    affects the accuracy of the matrix completion methods.
  \item illustrates that the matrix completion-based methods mis-predicts the
    users' top rated items because of the skewed distribution of ratings in the user-item
    rating matrix.
  \item shows that the false positives in \TOPN item recommendations generated by the matrix 
  	completion-based methods are not rated significantly low.  
  \item shows that the number of ratings an item has, i.e., item frequency, affect the accuracy of
    the matrix completion and the \TOPN item recommendations.
  %\item develops \TMF method, which leverages the insights derived in this chapter and outperforms the state-of-the-art \MF method for users and items with few ratings in real datasets. 
\end{enumerate}

\iffalse
Also, we show that the error in
predictions for such items increases further with the increase in the rank of
the low-rank models. 
Furthermore, we use these findings to develop \TMF which considers the number of
ratings received by an item or provided by a user to estimate the rating of the
user on the item. 
The exhaustive experiments on the real datasets demonstrate the effectiveness of
\TMF over the state-of-the-art \MF method for the users and the items with few
ratings.
\fi



 


