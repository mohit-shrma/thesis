\section{Conclusion} \label{ch:matcomp:conclusion}


In this work, we have investigated the performance of the matrix
completion-based low-rank models for estimating the missing ratings
in  user-item rating matrices having sparsity structure identical 
to real datasets. We showed in
Section~\ref{ch:matcomp:mf_accu} that the matrix completion-based methods because of the
presence of skewed distribution of entries in rating matrices in real
datasets fail to predict the missing entries accurately in the matrices. Also,
we learned that the items with high frequency are predicted more accurately than
the others. These findings imply that for a user, the unrated items which have more ratings 
in the matrix and are predicted high for the user, will form a better set of recommendations 
than the items with fewer ratings in the rating matrix.


Further, we saw in Section~\ref{ch:matcomp:mf_rank} that the errors in predictions due to
the skewed distribution of ratings in the user-item rating matrix 
affect the ranking performance of the matrix-completion based methods.
In particular, under the assumption that the rating that a user will provide to an item determines his
ranking preference, our results indicate that the items predicted at the top by matrix completion-based
methods miss a large number of true high rated items. In some datasets, the true
high rated items are missing even in the top $50\%$ of the predicted items for
the user.
However, the items that are predicted at the top for a user but are absent from
the true high rated items are present close to the true high rated items by the
user. Therefore, the ranking based on the predicted ratings is not severely
affected by false positives as it does not contain items that are significantly
low rated.
Additionally, we observed that the infrequent items, irrespective of whether 
they are true high rated or true low rated, are predicted low by the matrix
completion-based methods thereby appearing later in the ranking
of the items for recommendations.

\iffalse
We can further use these findings to improve existing matrix completion-based recommendation methods.
Also, it will be interesting to investigate how 
properties of ratings, e.g., diversity of the ratings, in the rating matrix
affect the matrix completion-based recommendation methods.
\fi

\iffalse
Additionally, we observed that the infrequent items irrespective of 
whether they are true high rated or true low rated are predicted low by the
matrix completion-based methods.
Since both the infrequent and the frequent items that are true low rated for
a user are predicted low, the \TOPN item
recommendations usually contain favorable items for the user.
\fi

\iffalse

\fi

\iffalse
In this work, we have investigated the performance of the matrix
completion-based low-rank models for estimating the missing ratings in real
datasets and its impact on the item recommendations. 
We showed in
Section~\ref{mf_accu} that the matrix completion-based methods because of the
presence of non-uniform randomly sampled entries in rating matrices in real
datasets fail to predict the missing entries accurately in the matrices. Also,
we learned that the items with high frequency are predicted more accurately than
the others. These findings imply that for a user, the unrated items which have more ratings 
in the matrix and are predicted high for the user, will form a better set of recommendations 
than the items with fewer ratings in the rating matrix.



We can further use these findings to improve existing matrix completion-based recommendation methods.
Also similar to item frequency, it will be interesting to investigate how other
properties of ratings, e.g., diversity of the ratings, in the rating matrix
affect the matrix completion-based recommendation methods.
\fi
