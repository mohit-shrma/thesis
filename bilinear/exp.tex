
\section{Experimental Evaluation} \label{ch:bilinear:experiments}

In this section we perform experiments to 
demonstrate the effectiveness of the proposed algorithm. 

\subsection{Datasets}
We used four datasets to evaluate the performance of \CF.

\CULEXP (\CUL)\footnote{http://citeulike.org/} aids researchers by allowing
them to add scientific articles to their libraries. For users of the \CUL, the
articles present in their library are considered as preferred articles i.e., $1$
in a preference matrix while rest are considered as implicit $0$ preferences. 

\MLHREXP (\MLHR) is the dataset described in~\cite{cantador11}. The ratings were
binarized by treating all ratings greater than $2$ as $1$ and below or equal 
to $2$ as $0$. The movie genres were used as the item's content. 
 
Amazon Books (\AMAZON) is a dataset collected from amazon about best-selling books
and their ratings. The ratings are binarized by treating all ratings greater
than equal to $3$ as $1$ and ratings below $3$ as $0$. Each is accompanied with a
description which was used as item's content.

Various statistics about these datasets are shown in Table~\ref{datasets_table}.
Also comparing the densities of the datasets we can see that the \MLHREXP dataset 
have significantly higher density than other dataset.

\BX (BX) dataset is extracted from Book Crossing
data~\cite{ziegler05www} such
that user has given at least four ratings and each book has received the same
amount of ratings. Description of these books were collected from Amazon using
ISBN and were used as item features.

For the \AMAZON, \CUL and \BX dataset, the words that appear in the item descriptions were
collected, stop words were removed and the remaining words were stemmed to
generate the terms that were used as the item features. All words that appear in
less than 20 items and all words that appear in more than 20\% of the items were
omitted. The remaining words were represented with TF-IDF scores. The item
feature matrix was normalized row-wise in datasets. 


%================================================================================


% \begin{table*}[thb]\footnotesize
%    \centering
%    %\tbl {Statistics for the datasets used for testing\label{datasets_table}}{%
%     \caption{Statistics for the datasets used for testing} \label{datasets_table}
%     %\begin{center}
    
%     %\vspace{-1pt}
   
%     \begin{threeparttable}
%     %\begin{scriptsize}
%     %\caption{Statistics for the 5 datasets used for testing}  
%     \begin{tabular}{
%       @{\hspace{2pt}}l@{\hspace{3pt}}
%       @{\hspace{2pt}}r@{\hspace{3pt}}
%       @{\hspace{2pt}}r@{\hspace{3pt}}
%       @{\hspace{2pt}}r@{\hspace{3pt}}
%       @{\hspace{2pt}}r@{\hspace{3pt}}
%       @{\hspace{2pt}}r@{\hspace{3pt}}
%       @{\hspace{2pt}}r@{\hspace{3pt}}
%       @{\hspace{2pt}}r@{\hspace{3pt}}
%     }
% 	%\toprule
% 	%ML100K
% 	%\toprule
% 	\hline
% 	\hline
% 	\textbf{Dataset}	& \textbf{users}	&	\textbf{items}		&	\textbf{features}	&	\textbf{preferences}	&	\textbf{prefs/user}	&	\textbf{prefs/item}	& \textbf{density}\\
% 	\hline
% 	\hline
% 	%\rule{-2pt}{3ex}
% 	\CUL  		& 	3,272 & 	21,508 	& 	6,359 	& 	180,622 	&	55.2 		& 	8.4 		&  	0.13\%	\\
%  	\MLHREXP 	&	  2,113 & 	10,109 	& 	20 	    & 	855,598 	& 404.9 	& 	84.6    &	  4.01\%	\\
% 	\hline
% 	\hline
	
% 	\end{tabular}

\begin{table*}[t]%\footnotesize
   \centering
   %\tbl {Statistics for the datasets used for testing\label{datasets_table}}{%
    \caption{Statistics for the datasets used for testing} \label{datasets_table}
    %\begin{center}
    
    %\vspace{-1pt}
   

    %\begin{scriptsize}
    %\caption{Statistics for the 5 datasets used for testing}  
    \begin{tabular}{
      @{\hspace{2pt}}l@{\hspace{3pt}}
      @{\hspace{2pt}}r@{\hspace{3pt}}
      @{\hspace{2pt}}r@{\hspace{3pt}}
      @{\hspace{2pt}}r@{\hspace{3pt}}
      @{\hspace{2pt}}r@{\hspace{3pt}}
      @{\hspace{2pt}}r@{\hspace{3pt}}
      @{\hspace{2pt}}r@{\hspace{3pt}}
      @{\hspace{2pt}}r@{\hspace{3pt}}
    }
  %\toprule
  %ML100K
  %\toprule
  \hline
  \textbf{Dataset}  & \textbf{users}  & \textbf{items}    & \textbf{features} & \textbf{preferences}  & \textbf{prefs/user} & \textbf{prefs/item} & \textbf{density}\\
  \hline
  %\rule{-2pt}{3ex}
  \MLHR  &   2,113 &   10,109  &   20      &   855,598   & 404.9   &   84.6    &   4.01\%  \\
  \CUL      &   3,272 &   21,508  &   6,359   &   180,622   & 55.2    &   8.4     &   0.13\%  \\
	\BX & 17,219 &	36,546 	& 	8,946 	& 	574,127 	& 	33.3 		& 	15.7 		&	0.09\%	\\
  \AMAZON		 & 13,097 	& 	11,077 	& 	5,766 	& 	175,612 	& 	13.4 		& 	15.9	&	0.12\%	\\ 
  \hline
  
  \end{tabular}

    %\end{scriptsize}
   % \end{center}
    %\vspace{-20pt}   	
    %\caption{Statistics for the 5 datasets used for testing}  \label{datasets_table}
\end{table*}

%---------------------------------------------------------------------------


\subsection{Comparison methods} \label{other_methods}
%TODO: comparison with latent factor based method
We compared \CF against non-collaborative personalized user modeling methods and
collaborative methods.

\begin{enumerate}
  \item \textbf{Cosine-Similarity (\COSIM)}
       This is a personalized user-modeling method.
        The preference score of user $u$ on target item $i$ is estimated using
        Equation~\ref{eq_est_true} by using \emph{Cosine Similarity} to compute
        similarity between the items.
  \item{\textbf{\CFLINEXP (\CFLIN)}}
        As mentioned before, this method~\cite{elbadrawy2015} learns personalized user model
        by using all the past preferences from users across the dataset. It
        outperformed other state of the art collaborative latent factor based methods e.g.,
        RLFM~\cite{agarwal09},  AFM~\cite{gantner10} by significant margin.
      %\item \textbf{\AFMI:}
  \item{\textbf{\RLFMI}}
        We used the Regression-based Latent Factor Modeling(\RLFM) technique
        implemented in LibFM~\cite{rendle12libfm} that accounts for inter-feature interactions. We
        used LibFM with SGD learning to obtain results.
\end{enumerate}




\subsection{Evaluation Methodology and Metrics}
For each dataset we split the corresponding user-item preference matrix  $R$ into three
matrices $R_{train}$, $R_{val}$ and $R_{test}$. $R_{train}$ contains a randomly
selected $60\%$ of the columns (items) of R, and the remaining columns were divided equally
among $R_{val}$ and $R_{test}$. Since items in $R_{test}$ and $R_{val}$ are not
present in $R_{train}$, this allows us to evaluate the methods for item cold-start
problems as users in $R_{train}$ do not have any preferences for items in
$R_{test}$ or $R_{val}$. 
The models are learned using $R_{train}$ and the best model is
selected based on its performance on the validation set $R_{val}$. The selected model
is then used to estimate the preferences over all items in $R_{test}$. For each
user the items are sorted in decreasing order and the first $n$ items are returned
as the \TOPN recommendations for each user. The evaluation metrics as described
later are computed using these \TOPN recommendation for each user. 

After creating the train, validation and test split, there might be some users who do
not have
any items in the validation or the test split. In that case we evaluate the performance on
the splits for only those users who have at least one item in the corresponding test split.
This split-train-evaluate procedure is repeated three times for each dataset and
the evaluation metric scores are averaged over these runs before being reported in
results.

We used two metrics to assess the performance of the various methods: Recall at
$n$ (Rec@n) and Discounted Cumulative Gain at $n$ (DCG@n).
Given the list of the \TOPN recommended items for user $u$, \RECN measures how
many of the items liked by $u$ appeared in that list, whereas the \DCGN measures
how high the relevant items were placed in the list. 
The \RECN is defined as 
\begin{equation*}
  REC@n = \frac{|\{\mbox{Items liked by user}\} \cap \{\mbox{Top-n 
  items}\}|}{|\mbox{Top-n items}|}
\end{equation*}
The \DCGN is defined as
\begin{equation*}
  DCG@n= \imp_1 + \sum_{p=2}^{n} \frac{\imp_p}{\log_2(p)},
\end{equation*} 
where the $\imp_p$ of the item with rank $p$ in the \TOPN list is
%
\begin{equation*}
\imp_p = \left\{
  \begin{array}{l l}
    1/n, & \quad \text{if item at rank $p \in R_{u,test}^+$}\\
    0, & \quad \text{if item at rank $p \notin R_{u,test}^+$}.
  \end{array} \right. 
  \end{equation*} 
The main difference between \RECN and \DCGN is that \DCGN is sensitive to the rank of the 
items in the \TOPN list. Both the \RECN and the \DCGN are computed for each user and then 
averaged over all the users. 
%
%We also computed the Precision at $n$ metric (Prec@$n$); however, the ranking of the methods under Prec@$n$ was the same as their ranking under \RECN and for this reason we only report the \RECN values.

\subsection{Model Training} \label{evaluation}
\CF's model parameters are estimated using training set $R_{train}$ and
validation set $R_{val}$. After each
major SGD iteration of Algorithm \ref{alg-bpr} we compute
the \RECN on validation set and save the current model if current \RECN is 
better than those computed in previous iterations. The learning process ends when
the optimization objective converges or no further improvement in validation recall 
is observed for $10$ major SGD iterations. At the end of learning process we return the model that achieved the best \RECN on
the validation set.

To estimate the model parameters of $\CF_{bpr}$, we draw samples
equal to the number of preferences in $R$ for each major SGD iteration. Each sample
consists of a user, an item preferred by user and an item not preferred by user.
If a dataset does not contain items not preferred by user then we sample from
items for which his preference is not known.

