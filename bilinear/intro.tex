\section{Introduction} \label{ch:bilinear:introduction}

\TOPN recommender systems are used to identify from a large pool of items those $n$
items that are the most relevant to a user and have become an essential
personalization and information filtering technology. They rely on the historical
preferences that were either explicitly or implicitly provided for the items and
typically employ various machine learning methods to build \emph{content-agnostic}
predictive models from these preferences. However, when new items are introduced into
the system, these approaches cannot be used to compute personalized recommendations,
because there are no prior preferences associated with those items.
%
As a result, the methods used to recommend new items, referred to as (item)
\emph{cold-start} recommender systems, in addition to the historical information,
take into account the characteristics of the items being recommended; that is, they
are \emph{content aware}. The items' characteristics are typically captured by a set
of domain-specific features. For example, a movie may have features like genre,
actors, and plot keywords; a book typically has features like content description and
author information. These item features are intrinsic to the item and as such they do
not depend on historical preferences. 
%These item features are available as soon as the new items are introduced in the
%inventory and can be used for recommender systems in absence of historical records
%on the interactions among users and these items.





%TODO: move this to relevant work section
\iffalse 
Besides the non-collaborative models, another important group of techniques used in cold-start recommender systems are collaborative methods~\cite{r1,r2,r3,r9}, which are based on latent factors. These models leverage the knowledge
available from all the users and their past preferences. However, these latent
factor methods don't perform well when sparsity of dataset grows, i.e., items have limited number of preferences on average~\cite{r29}.
\fi

%cite some references
In this work, we extend \CFLIN, previously described in Section~\ref{ch:related}, 
in order to account for interactions between the
different item features. We believe that such interactions are
important and quite common. For example, in an e-commerce website, the items
that users tend to buy are often designed to go well with previously purchased
items (e.g., a pair of shoes that goes well with a dress). The set of features
describing items of different type will be different (e.g., shoe material and
fabric color) and as such a linear model can not learn from the data that for
example a user prefers to wear leather shoes with black clothes. Being able to model such
dependencies can lead to item cold-start recommendation algorithms that achieve
better performance.

Towards this goal, we present a method called \CFEXPB (\CF) that uses bilinear
models to capture pairwise dependencies between the features. Like \CFLIN, \CF
estimates a user-specific similarity function by linearly combining a small
number of global similarity functions. However, unlike \CFLIN's linear global
similarity functions, \CF's global similarity functions are bilinear.
A challenge associated with such bilinear models is that the number of
parameters that needs to be estimated becomes quadratic on the dimensionality of
the item's feature space, which is problematic given the very sparse training
data. \CF overcomes this challenge by assuming that the pairwise relations can
be modeled as a combination of a linear component and a low rank component. The
linear component allows it to capture the direct relations between the features
whereas the low rank component allows it to capture the pairwise
relations. 
The parameters of these models are
estimated using stochastic gradient descent and a ranking loss function based on
Bayesian personalized ranking (BPR) that optimizes the area under the receiver
operating characteristic curve.

We performed extensive empirical studies to evaluate the performance of the proposed
\CF on two benchmark datasets and compared it against state-of-the-art models for
item cold-start recommendation. In our results \CF optimized using the BPR loss function can
improve upon other methods in terms of recommendation quality, especially for
datasets in which there are considerable historical data for each item and
datasets whose items are described by relatively small number of features.

%The chapter here is a generalized extension of work~\cite{r43} I did at Samsung Research America during my summer internship in 2014. 
This chapter is a generalized extension of Feature-based factorized  Bilinear Similarity Model (FBSM)~\cite{r43}.

\iffalse
Rest of the chapter is organized as follow: 
%Section~\ref{} %$2$
%defines the notations and
%definitions that are used in the chapter. 
%Section~\ref{} %$3$
%discusses the related work.
Section~\ref{ch:bilinear:method} %$4$
describes \CF and optimization methods that are used to estimate
model parameters. Section~\ref{ch:bilinear:experiments}  %$5$
shows the evaluation methodology and section~\ref{ch:bilinear:results} %$6$
discusses the results. 
Section~\ref{ch:bilinear:conclusion} %$7$ 
concludes the work in the chapter.
\fi