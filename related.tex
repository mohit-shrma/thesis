\chapter{Background and Related Work}
\label{ch:related}

%============================ General recommendations related work: CF and neighborhood intro =====
Recommender systems employ different algorithms to generate recommendations. These algorithms fall into
two different classes: content-based methods~\cite{lops2011content,pazzani2007content} and collaborative filtering-based methods~\cite{herlocker1999algorithmic}. Content-based methods rely on the attributes of the users and the items to generate recommendations. Collaborative filtering-based
methodsmake use of the user preferences available in the form explicit ratings or implicit feedback. These
methods utilize the user or item co-rating information to estimate the user preferences over the items.
Collaborative filtering-based approaches can be further divided into two categories, i.e., neighborhood-based
methods~\cite{shardanand1995social, konstan1997grouplens, SarwarKarypis01, deshpande2004item} and model-based or latent factor-based methods~\cite{koren2008factorization,Koren2009,koren2010collaborative}.  The neighborhood-based methods learn the user or the
item neighborhood based on the co-rating information to generate the recommendations.  The model-based approaches
learn a model, i.e., the user and the item latent factors,  from the rating data and use it to generate the
recommendations. Next, we will discuss some of the work that is relevant to this thesis.







%============================ Matrix Completion ==================================================
\paragraph{Matrix Completion}
The state-of-the-art methods for recommendations are based on matrix
completion, and most of them involve factorizing the user-item rating
matrix~\cite{Koren2009, koren2008factorization, hu2008collaborative}.
%In this work, our focus is on analyzing the performance of the matrix
%factorization-based matrix completion approach. 
The \MF (MF) method assume that the
user-item rating matrix is low-rank and can be computed as a product of two
matrices known as the user and the item latent factors. 
If for a user $u$, the vector $\bm{p_u} \in \mathbb{R}^f$ denotes the $f$ dimensional
user's latent factor and similarly for the item $i$, the vector 
$\bm{q_i} \in \mathbb{R}^f$ represents the $f$ dimensional item's latent factor, then the
predicted rating for the user $u$ on the item $i$, i.e., $\hat{r}_{u,i}$ is given by

\begin{equation} \label{mf_eq}
  \begin{split}
    \hat{r}_{u,i} = \bm{p_u}\bm{q_i}^T.
  \end{split}
\end{equation}


The user and the item latent factors are estimated by minimizing a regularized
square loss between the actual and predicted ratings 
\begin{equation} \label{mf_obj}
  \begin{aligned}
    \minimize_{P, Q} & &\frac{1}{2}\sum_{r_{ui} \in R}
    \left(r_{ui} - \bm{p}_u\bm{q}_i^T \right)^2+ \frac{\beta}{2}
    \left(||P||_F^2 + ||Q||_F^2 \right),
  \end{aligned}
\end{equation}
where the matrices $P \in \mathbb{R}^{m \times f}$ and $Q \in\mathbb{R}^{n \times f}$ 
contains latent factors of the users and the items
respectively. The parameter $\beta$ controls the Frobenius norm regularization
of the latent factors to prevent overfitting.

In a separate body of work~\cite{CandesTao2010, CandesRecht09}, it has been shown that in order to complete a $n \times n$ matrix of rank $r$
accurately by matrix completion-based methods, $O(nr \log(n))$ entries should be sampled uniformly at random from the
matrix.

\iffalse
In real-world recommendation systems, it is not possible for a user to indicate
his or her preferences over all the available items thereby leading to missing
ratings in the user-item rating matrix. There has been some work that assumes
all ratings are \emph{missing not at random} (MNAR) and models the missing data to improve the
generated recommendations~\cite{marlin2012collaborative, pan2008one,
hernandez2014probabilistic, Steck2010MNAR, Steck2013EvalRec, lim2015top,
KimRecsys2014}.

However, in our work, we focus on investigating how
do the missing ratings, i.e., both \emph{missing at random} (MAR) and \emph{missing not at
random} (MNAR), or specifically how does the presence of few ratings for the
users and the items affect the matrix completion-based methods. We use our analysis
to develop a matrix completion-based approach to improve recommendations for the
users who have provided few ratings on items or for the items that have
received few ratings from the users.
\fi

There has been some work on locality-based matrix completion  methods which
assume that different parts of the user-item rating matrix can be approximated
accurately by different low-rank models~\cite{lee2013local,
lee2014local,chen2015wemarec}. The complete user-item rating matrix
can be approximated as a weighted sum of these individual low-rank models. 

\iffalse
In
our work, we estimate a single low-rank model, rather than multiple different
low-rank models, that adaptively select a subset of ranks for a rating by a
user on an item by considering the number of ratings provided by the user and
the number of ratings received by the item.
\fi



% =============== cold-start ===============================================
\paragraph{Cold-start Item Recommendations}
The prior work to address the cold-start item recommendation can be divided 
into \emph{non-collaborative} user personalized models 
and \emph{collaborative} models. The non-collaborative models generate recommendations 
using only the user's past
interaction history and the collaborative models combine information from
the preferences of different users. 

Billsus and Pazzani~\cite{billsus99} developed one of the first user-modeling
approaches to identify relevant new items. In this approach they used the users'
past preferences to build user-specific models to classify new items
as either ``relevant'' or ``irrelevant''. The user models were built using
item features e.g., lexical word features for articles. Personalized 
user models~\cite{rodriguez01} were also used to classify news feeds by modeling short-term user needs 
using text-based features of items that were recently viewed by user and long-term needs were
modeled using news topics/categories. Banos~\cite{banos06}
used topic taxonomies and synonyms to build high-accuracy 
content-based user models. 

Recently collaborative filtering techniques using latent
factor models have been used to address cold start item recommendation
problems. These techniques incorporate item features in their factorization
techniques. Regression-based latent factor models (RLFM)~\cite{agarwal09} is a
general technique that can also work in item cold-start scenarios. 
RLFM learns a latent factor representation of the preference matrix
in which item features are transformed into a low dimensional space using regression.
This mapping can be used to obtain a low dimensional representation of the cold-start
items. User's preference on a new item is estimated by a dot product of
corresponding low dimensional representations. The RLFM model was further
improved by applying more flexible regression models~\cite{zhangrecsys11}.
AFM~\cite{gantner10} learns item attributes to latent feature mapping
by learning a factorization of the preference matrix
into user and item latent factors $R = PQ^T$. A mapping function is then learned
to transform item attributes to a latent feature representation i.e., $R = PQ^T =
PAF^T$ where $F$ represents items' attributes and $A$ transforms the items' attributes 
to their latent feature representation. 
%Bayesian Personalized Ranking (BPR) Matrix Factorization
%\cite{r6} was used to learn user and item latent factors i.e $P$ and $Q$
%respectively. BPR loss function was also used to learn the mapping $A$.  

%can add that AFM performance degrades on using high dimensional features as
%reported in Asmaa's paper

\CFLINEXP (\CFLIN)~\cite{elbadrawy2015} learns a personalized user
model by using historical preferences from all users across the dataset. In
this model for each user an item similarity function is learned, which is a
linear combination of user-independent similarity functions known as
\textit{global similarity functions}. Along with these global similarity
functions, for each user a personalized linear combination of these global
similarity functions is learned. It is shown to outperform both RLFM and AFM
methods in cold-start \TOPN item recommendations.


%%%%%%%%%%%%%%%%%%%%%%%%%%%%%%%%%%%%%%%%%%%%%%%%%%%%%%%%%%%%%%
Predictive bilinear regression models~\cite{chu09www} belong to the feature-based
machine learning approach to handle the cold-start scenario for both users and
items. Bilinear models can be derived from Tucker family~\cite{tucker66}. They have
been applied to separate ``style'' and ``content'' in images~\cite{tenenbaum00}, to
match search queries and documents~\cite{wu13jmlr}, to perform semi-infinite stream
analysis~\cite{sun06kdd}, and etc. Bilinear regression models try to exploit the
correlation between user and item features by capturing the effect of pairwise
associations between them. Let $\bm{x}_i$ denotes features for user $i$ and $\bm{x}_j$
denotes features for item $j$, and a parametric bilinear indicator of the
interaction between them is given by $s_{ij} = \bm{x}_iW\bm{x}_j^T$ where $W$ denotes
the matrix that describes a linear projection from the user feature space onto
the item feature space. The method was developed for recommending cold-start
items in the real time scenario, where the item space is small but dynamic
with temporal characteristics. In another work~\cite{park09recsys}, authors proposed to
use a pairwise loss function in the regression framework to learn the matrix
$W$, which can be applied to scenario where the item space is static but
large, and we need a ranked list of items.







% =============== learning from sets =======================================
\paragraph{Learning From Sets of Items}
There has been little published work on using set-level ratings to improve the
accuracy of item-level recommendations. The one exception is a recent study in which relative
preference information on different groups of items was collected during a new user
signup process and these preferences were then used to assign a user to a set of
pre-built recommendation profiles~\cite{r53}. 
This approach significantly reduced the time required to learn the user's
preferences in order to generate recommendations for the new user.
The principal difference from this 
approach is that in this thesis we try to model the user behavior that determines
his/her estimated rating on a set and then use that to develop fully
personalized recommendation methods that are not limited to new users.

In addition, there has been some work that  has focused on recommending
lists of items or bundles of items. For example, recommendation of music
playlists~\cite{r55,moore2012learning,aizenberg2012build}, travel
packages~\cite{interdonato2013versatile,r54,liu2011personalized,xie2011comprec}, reading
lists~\cite{r56} and recommendation of lists under user specified budget
constraints~\cite{xie2010breaking,BenouaretRecsys16}.
However, this research is not directly related to the problems explored in this
thesis because our focus is on learning the user's ratings on items in lists from
the ratings that the user provided on these lists.

Another relevant work is the problem of energy disaggregation~\cite{hart1992nonintrusive}, 
which refers to the task
of separating the energy signal of a building into the energy signals of
individual appliances that reside in the building. Disaggregated energy
consumptions are used to provide feedback to consumers,  forecast demands,
design energy incentives and detect appliances'
malfunction~\cite{froehlich2011disaggregated,darby2006effectiveness}. Similar to
the idea of energy disaggregation, in this thesis, we try to separate a user's rating on a set of
items into the users' ratings on items in the set and generate item
recommendations for the user.

The researchers have also investigated how different aspects, e.g., rating
questions~\cite{bellogin2014magic}, reference
points~\cite{adomavicius2011recommender,cosley2003seeing,nguyen2013rating}, and
contextual factors~\cite{Winoto2010RUM}, can influence a user when
elicited to provide a rating on an item. In this thesis, we have investigated how
does the user provides a rating on a set of items and used the derive insights to develop
collaborative filtering-based methods to predict the rating for an individual
item in the set.









