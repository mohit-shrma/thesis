\section{Introduction} 
\label{ch:tmf:intro}


The matrix completion-based methods, e.g., \MF (MF)~\cite{Koren2009,
koren2008factorization, hu2008collaborative}, are the state-of-the-art
collaborative filtering methods that use users' historical preferences 
over items to generate recommendations. The ratings provided by users over
the items can be viewed as a matrix whose rows represent the users, columns
denote the items, and entries are the ratings provided by the users over the
items.  There exists a small number of attributes that describe the items, and
a user's rating depends on how the user values those attributes. This makes the
user-item rating matrix low-rank, and the number of attributes determines the
rank of the matrix. The attributes of the items and the weights provided by
a user over these attributes are often referred as the items' latent factors
and the users' latent factors respectively. MF estimates the user-item rating
matrix as the product of the user latent factors and the item latent factors.  


In practice, there are few attributes that are responsible for a large
portion of the rating provided by a user on an item and the remaining rating
can be explained by other attributes. However, certain users have provided
ratings to few items, and some items have received few ratings from the users
thereby these users or items may not have sufficient ratings to estimate weights
for all the attributes accurately. The inaccuracy in the estimation of weights
for these users and items can affect the predicted ratings and hence affect
the generated recommendations. Therefore, the recommendations for these users
and items with few ratings may improve by focusing on few attributes that
are responsible for a significant portion of the rating and can be estimated
accurately by matrix completion-based methods.


This chapter investigates how does the performance of the matrix
completion-based methods changes with the number of ratings that a
user or an item has, and shows that the users or the items with few
ratings tend to have low accuracy. Additionally, we show that the error
in predictions for such users or items increases further with the 
increase in rank of the low-rank matrix completion-based methods. 
Furthermore, we use these findings to develop \TMF which considers the number of
ratings received by an item or provided by a user to estimate the rating of the
user on the item. 
The exhaustive experiments on the real datasets demonstrate the effectiveness of
\TMF over the state-of-the-art MF method for the users and the items with few
ratings.





