Recommender systems are widely used to recommend the most appealing items to
users.
In this thesis, we focus on analyzing the accuracy of the state-of-the-art matrix
completion-based recommendation methods and develop methods to model users' 
preferences to address different problems that arise in recommender systems.
 
Collaborative filtering-based methods are widely used to generate item recommendations to the user. The low-rank matrix completion method is the state-of-the-art collaborative filtering method.  
We will show that the accuracy and the ranking performance of matrix completion-based
methods are affected by the skewed distribution of ratings in the user-item rating matrix.
Additionally, we will illustrate that the number of ratings an item has positively
correlates with the prediction accuracy and the ranking performance of the matrix
completion approach for the item.
Furthermore, we show that the users or the items that are present in the tail, i.e.,
those having few ratings in real datasets, may not have sufficient ratings to
estimate the low-rank models accurately by matrix completion approach. We use
these insights to develop \TMF, a matrix completion-based approach that outperforms
the state-of-the-art matrix completion method for the users and the items in the tail.


Since for new items we do not have any prior preferences from existing users, it is hard
to recommend these items to the users.  We can use non-collaborative methods that rely
on similarities between the new item and the items preferred by a user in the past.
However, these methods consider the item features independently and ignore the
interactions among the features of the items while computing the similarities. 
Modeling the interactions among features can provide more information towards the
relevance of an item in comparison to the scenario when the features are considered
independently.
We develop a new method called \CFEXPB (\CF), that uses all available information
across users to capture these interactions among features and learns a \emph{low-rank}
user personalized \emph{bilinear} similarity model for \TOPN recommendation
of new items.


In addition to providing ratings over individual items, the users can also provide
ratings on sets of items. A rating provided by a user on a set of items conveys some
preference information about the items in the set and enables us to acquire a user's
preferences for more items that the number of ratings that the user provided. Moreover,
users may have privacy concerns and hence may not be willing to indicate their preferences
on individual items explicitly but may be willing to provide a rating to a set of items,
as it provides some level of information hiding.
We will investigate how do users' item-level preferences relate to their set-level preferences
and how collaborative filtering-based models can take advantage of set-level preferences
to generate item recommendations. 
%We will introduce collaborative filtering-based methods
%that explicitly model the user behavior of providing ratings on sets of items and can be
%used to recommend items to users.





