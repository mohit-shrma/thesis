\section{Introduction}

%Today consumers are overwhelmed with the enormous volume of products that is
%available for their consumption. 

Recommender systems help consumers by providing suggestions that are
expected to satisfy their tastes. They are successfully  deployed in several domains
such as e-commerce (e.g., Amazon, Ebay), multimedia content providers (e.g., Netflix,
Hulu) and mobile app stores (e.g., Apple, Google Play).
Collaborative filtering~\cite{r30,koren2009matrix} which takes advantage of users' past preferences
to suggest relevant items, is one of the key methods
used by recommender systems.


Most collaborative filtering approaches rely on past preferences provided by
users on individual items.
An additional source of preferences is the user's preferences on sets of items.
Example of such set-level ratings includes
ratings on song playlists, music albums, reading lists,
and watchlists. 
A rating provided by the user on a set of items conveys some information about the
user's preference on each of the set's items and, as a result, it is a
mechanism by which some information about user's preferences can be acquired for
many items.
At the same time, due to privacy concerns, users that are not willing to
explicitly reveal their true preferences on individual items may provide a single
rating to a set of items, since it provides some level of information hiding.
%The acquisition of the users' preferences over individual items can be
%time-consuming, as users have to indicate their preferences on items
%sequentially~\cite{drenner2008crafting,Rashid02IUI}.

This chapter investigates two questions related to using set-level preferences
in recommender systems. 
First, how users' item-level ratings relate to the
ratings that they provide on a set of items. Second, how collaborative
filtering-based methods
can take advantage of such set-level ratings towards making item-level rating
predictions.


To answer the first question, we collected ratings on sets
of movies from users of \ML, a popular online movie recommender
system\footnote{www.movielens.org}. 
Our analysis of these ratings leads to two key findings. First, for the majority
of the users, the rating provided on a set can be accurately approximated by the
average rating that they provided on the set's constituent items. Second, there
is a considerable user population that tends to consistently either over- or
under-rate the set, especially for sets that contain items on which the user's
item-level ratings are diverse.
Using these insights, we developed different  models that can predict
a user's rating on a set of items as well as on individual items. 
Furthermore, these methods can use ratings on both the sets and the items and
lead to better results for the users that have either both or only one type of
ratings.
These methods solve
these problems in a coupled fashion by estimating models to predict the item-level
ratings and by estimating models that combine these individual ratings to derive
set-level ratings.



The key contributions of the work are the following:
\begin{enumerate}[(i)]
\item collection and analysis of a dataset that contains users' ratings both on
individual items and on various sets containing these items;
%
\item introduction of \emph{Variance Offset Average Rating Model} (\VO) and
\emph{Extremal Subset Average Rating Model} (\ES) to
model a user's consistency to over- or under-rate the set of items as a function of his/her ratings on the set's constituent items; and
%
\item development of collaborative filtering-based methods that take
advantage of \VO and \ES in order to estimate users' preferences on sets of items as well as
on individual items.
\end{enumerate}

%\begin{enumerate}
%  \item Analysis of the ratings acquired from real users on sets of items. 
%  \item Development of matrix factorization based approaches to estimate the
%    users' preferences on items from ratings on sets of items or from 
%    ratings on both sets of items and items in the sets. 
%  \item Release of the collected dataset to the community for research
%    purposes..
%\end{enumerate}

\iffalse
The rest of the chapter is organized as follows. 
Section~\ref{ch:lfs:dataset} describes the dataset creation process
along with the analysis of the set ratings in relation to the users' ratings on
their constituent items. 
Section~\ref{ch:lfs:lfs_method} presents the methods that we developed to estimate the
item-level models from the set ratings. 
Section~\ref{ch:lfs:exp_eval} provides information about the evaluation methodology. 
Section~\ref{ch:lfs:results} presents the results of the experimental evaluation. Finally, 
Section~\ref{ch:lfs:conclusion} provides some concluding remarks.
\fi


