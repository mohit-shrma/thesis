\section{Conclusion} \label{ch:lfs:conclusion}
In this work, we studied how users' ratings on sets of items relate to their
ratings on the sets' individual items. 
We collected ratings from active users of \ML on sets of movies and based
on our analysis we developed collaborative filtering-based models that try to explicitly model the
users' behavior in providing the ratings on sets of items.
Through extensive experiments on synthetic and real data, we showed that the
proposed methods can model the users' behavior as seen in the real data and
predict the users' ratings on individual items.

\iffalse
For future work, it will be interesting to study how do the performance of the proposed
approaches vary with the different number of items in sets and how do they
perform when instead of having a fixed number of items in sets, the sets contain
a varied number of items in sets.
Also, the performance of the model could be improved by modeling
temporal effects on the ratings and by using side-information like genres or
other movie metadata. Finally, it will be interesting to investigate if similar to the diversity of
ratings in the set there exists other properties at item- or set-level
that can affect a user's ratings on sets of items.
\fi

