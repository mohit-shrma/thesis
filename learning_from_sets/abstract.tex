Most of the existing recommender systems use the ratings provided by users on
individual items.
An additional source of preference information is to use the ratings that users provide on sets of items.
The advantages of using preferences on sets are two-fold. First, a
rating provided on a set conveys some preference information about each of the
set's items, which allows us to acquire a user's preferences for more items that
the number of ratings that the user provided.
Second, due to
privacy concerns, users may not be willing to reveal their preferences on
individual items explicitly but may be willing to provide a single rating to a
set of items, since it provides some level of information hiding. This chapter
investigates two questions related to using set-level ratings in
recommender systems. First, how users' item-level ratings relate to their
set-level ratings. Second, how collaborative filtering-based models for
item-level rating prediction can take advantage of such set-level ratings. 
We have collected set-level ratings from active users of \ML
on sets of movies that they have rated in the past. 
Our analysis of these ratings shows that though the majority of the users
provide the average of the ratings on a set's constituent items as the rating on
the set, there exists a significant number of users that tend to
consistently either under- or over-rate the sets.
We have developed collaborative filtering-based
methods to explicitly model these user behaviors that can be used to recommend
items to users. 
Experiments on real data and on synthetic data that resembles the under- or
over-rating behavior in the real data, demonstrate that these models can recover the
overall characteristics of the underlying data and predict the user's ratings on
individual items.