\chapter{Conclusion}
\label{ch:conclusion}


\section{Thesis Summary}
Recommender systems are widely used to recommend relevant products to the users. They help a user by identifying few
relevant products from a catalog containing a large number of products and thus help the user by providing
personalized suggestions to the user. Recommendations are typically generated by using either content-based
or collaborative filtering-based methods. Content-based methods rely on attributes of users or items to generate
recommendations, and collaborative filtering-based methods rely on explicit or implicit preferences provided by the
users over items. Furthermore, collaborative filtering-based methods are divided into two classes,
i.e.,  neighborhood-based and latent factor-based methods. 
Neighborhood-based methods identify the user and item neighborhoods based on co-rating data to generate recommendations,
and matrix completion-based methods learn low-rank models from the data to generate recommendations.


In this thesis, we have investigated how the accuracy and the ranking performance of the matrix completion-based approaches are affected by the skewed distribution of ratings in the user-item rating matrices. 
Furthermore, we have model user preferences in scenarios where standard recommendation methods can not be applied.  We have developed methods to recommend new items, i.e., cold-start item recommendations and to leverage user preferences over sets of items to generate item recommendations.



\subsection*{Accuracy of matrix completion methods}
Matrix completion is the state-of-the-art collaborative filtering method
and is widely used to generate recommendations.
In this thesis, we investigated the effect of the skewed distribution of
ratings,  as found in real datasets, on the accuracy and the ranking
performance of matrix completion. We showed that skewed distribution affects
the accuracy of matrix completion and the item with high frequency are
predicted more accurately than the others. Additionally, we found that the
items predicted at the top by matrix completion miss a significant number of
true high-rated items. 
Furthermore, the ranking based on the predicted ratings is not severely
affected by false positives as the items that are predicted at the top for a
user but are absent from the true high rated items are present close to the
true high rated items by the user. Also, we saw that the infrequent items are predicted low by the matrix completion-based methods thereby appearing later in the ranking of the items for recommendations.



\subsection*{Truncated Matrix Factorization (\TMF)}
In practice, few item attributes determine a significant portion of the user rating and the leftover portion
of the rating is determined by other attributes. The item attributes and the user weights over these attributes
are known as the item latent factors and the user latent factors respectively. In this thesis, we showed that
in real datasets,  some users or items may not have sufficient ratings to estimate all the user and the item
latent factors accurately.
We developed \TMF, a matrix completion-based method which considers the number of ratings of a user and an
item to estimate the user's rating on the item. The exhaustive experiments on real datasets illustrate that
the \TMF method outperforms the state-of-the-art MF method for the task of rating prediction for the users
and the items with few ratings.



\subsection*{\CFEXPB}
Since we do not have any prior preferences for new items, the standard collaborative filtering methods can
not be used to recommend the new or \emph{cold-start} items. The non-collaborative methods that rely on
similarities between the new item and the items preferred by a user in the past can be used for cold-start
item recommendations. However, these non-collaborative methods ignore the interaction among features and
consider them independently while computing similarities. In this thesis, we presented \CFEXP (\CF) for
cold-start item recommendations. \CF captures the interaction among the item features and leverages the
information available from all the users to recommend new items. 
The extensive experiments on real dataset show that \CF can perform better than other methods in terms of
recommendation quality, especially in datasets that have relatively small number of features and a
considerable number of ratings for existing items.  



\subsection*{Learning from Sets of Items in Recommender Systems}
An additional source of information in recommender systems can be the ratings provided by the users on sets
of items. For example,  the users can provide ratings on music albums, song playlists, and reading lists. 
A preference provided by a user on a set of items indicates some information about the user's preference for
individual items in the set. Additionally, due to privacy concerns, users may not be willing to indicate their
preferences for individual items but may provide a rating to a set of items as it provides some level of
information hiding. In this thesis, we have investigated how a user's rating on a set of items relates to
individual item-level ratings and developed collaborative filtering methods that can use the set-level ratings
to generate item recommendations. 
The experiments on the real and the synthetic datasets show that the developed methods can recover the
characteristics of underlying data and can be used for item recommendations.





\section{Future research directions}
The problems explored and methods presented in this thesis can be further extended in multiple future directions. It will be interesting to investigate the effect of different properties of ratings, e.g., diversity of ratings, in the user-item rating matrix on the performance of the matrix completion-based recommendation methods.
We can also leverage the derived insight in Section~\ref{ch:tmf:freqanal}, i.e., only fewer dimensions of latent factors are estimated accurately for users or items with few ratings,  to modify
existing locality-based  matrix completion methods~\cite{lee2013local,
lee2014local,chen2015wemarec} by using lower ranks for
the sparse part and higher ranks for the dense part of the user-item rating
matrix.
Similar to \TMF, we may be able to improve other latent factor-based methods that may suffer from inaccuracy due to insufficient data, e.g.,  Factorization Machines~\cite{rendle2010factorization,rendle12libfm} and Word2Vec~\cite{mikolov2013distributed}.

Furthermore, we can improve the usage of preferences over sets of items by modeling temporal effects on the ratings and by using side-information like genres or other movie metadata. Also, it will be interesting to investigate if similar to the diversity of ratings in the set there exists other properties at the item- or set-level that can affect a user's ratings on sets of items.
Moreover, a user may rate the set of items independent of what is his preference for an individual item and instead rate the set depending on how does he perceive the set as a whole. In this scenario, the items in a set can complement each other and thereby receive a more favorable rating from the user. On the contrary, it could be possible that items in a set compete with each other and thus receive a more critical rating on the set. Thus, modeling the synergy and the competition among items in a set can further improve the estimation of the user preferences over sets and items.







