\chapter{Conclusion}
\label{ch:conclusion}


\section{Thesis Summary}
Recommender systems are widely used to recommend relevant products to the users. They help a user by identifying few
relevant products from a catalog containing a large number of products and thus help the user by providing
personalized suggestions to the user. Recommendations are typically generated by using either content-based
or collaborative filtering-based methods. Content-based methods rely on attributes of users or items to generate
recommendations, and collaborative filtering-based methods rely on explicit or implicit preferences provided by the
users over items. Furthermore, collaborative filtering-based methods are divided into two classes,
i.e.,  neighborhood-based and latent factor-based methods. 
Neighborhood-based methods identify the user and item neighborhoods based on co-rating data to generate recommendations,
and matrix completion-based methods learn low-rank models from the data to generate recommendations.


In this thesis, we have investigated how the accuracy and the ranking performance of the matrix completion-based approaches are affected by the skewed distribution of ratings in the user-item rating matrices. 
Furthermore, we have model user preferences in scenarios where standard recommendation methods can not be applied.  We have developed methods to recommend new items, i.e., cold-start item recommendations and to leverage user preferences over sets of items to generate item recommendations.



\subsection*{Accuracy of matrix completion methods}
Matrix completion is the state-of-the-art collaborative filtering method
and is widely used to generate recommendations.
In this thesis, we investigated the effect of the skewed distribution of
ratings,  as found in real datasets, on the accuracy and the ranking
performance of matrix completion. We showed that skewed distribution affects
the accuracy of matrix completion and the item with high frequency are
predicted more accurately than the others. Additionally, we found that the
items predicted at the top by matrix completion miss a significant number of
true high-rated items. 
Furthermore, the ranking based on the predicted ratings is not severely
affected by false positives as the items that are predicted at the top for a
user but are absent from the true high rated items are present close to the
true high rated items by the user. Also, we saw that the infrequent items are predicted low by the matrix completion-based methods thereby appearing later in the ranking of the items for recommendations.



\subsection*{Truncated Matrix Factorization (\TMF)}
In practice, few item attributes determine a significant portion of the user rating and the leftover portion of the rating is determined by other attributes. The item attributes and the user weights over these attributes are known as the item latent factors and the user latent factors respectively. In this thesis, we showed that in real datasets,  some users or items may not have sufficient ratings to estimate all the user and the item latent factors accurately.
We developed \TMF, a matrix completion-based method which considers the number of ratings of a user and an item to estimate the user's rating on the item. The exhaustive experiments on real datasets illustrate that the \TMF method outperforms the state-of-the-art \MF method for the task of rating prediction for the users and the items with few ratings.




\subsection*{\CFEXPB}






\subsection*{Learning from Sets of Items in Recommender Systems}







\section{Future research directions}






